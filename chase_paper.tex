
\documentclass[sigconf]{acmart}
%%
%% \BibTeX command to typeset BibTeX logo in the docs
\AtBeginDocument{%
  \providecommand\BibTeX{{%
    Bib\TeX}}}

%% Rights management information.  This information is sent to you
%% when you complete the rights form.  These commands have SAMPLE
%% values in them; it is your responsibility as an author to replace
%% the commands and values with those provided to you when you
%% complete the rights form.
\setcopyright{acmlicensed}
\copyrightyear{2026}
\acmYear{2026}
\acmDOI{XXXXXXX.XXXXXXX}
%% These commands are for a PROCEEDINGS abstract or paper.
\acmConference[Conference acronym 'XX]{Make sure to enter the correct
  conference title from your rights confirmation email}{June 03--05,
  2018}{Woodstock, NY}
%%
%%  Uncomment \acmBooktitle if the title of the proceedings is different
%%  from ``Proceedings of ...''!
%%
%%\acmBooktitle{Woodstock '18: ACM Symposium on Neural Gaze Detection,
%%  June 03--05, 2018, Woodstock, NY}
\acmISBN{978-1-4503-XXXX-X/2018/06}
\usepackage{stfloats}

%%
%% Submission ID.
%% Use this when submitting an article to a sponsored event. You'll
%% receive a unique submission ID from the organizers
%% of the event, and this ID should be used as the parameter to this command.
%%\acmSubmissionID{123-A56-BU3}


%%
%% end of the preamble, start of the body of the document source.
\begin{document}

%%
%% The "title" command has an optional parameter,
%% allowing the author to define a "short title" to be used in page headers.
\title{Low-Cost Roadside Diesel Emission Monitoring with LSTM-Based Sensor Data Enhancement}

%%
%% The "author" command and its associated commands are used to define
%% the authors and their affiliations.
%% Of note is the shared affiliation of the first two authors, and the
%% "authornote" and "authornotemark" commands
%% used to denote shared contribution to the research.
\author{Zachary Driskill}
\affiliation{%
  \institution{Brigham Young University}
  \city{Provo}
  \state{Utah}
  \country{USA}
}
\email{zadriskill@gmail.com}

\author{Lars Th{\o}rv{\"a}ld}
\affiliation{%
  \institution{The Th{\o}rv{\"a}ld Group}
  \city{Hekla}
  \country{Iceland}}
\email{larst@affiliation.org}

\author{Valerie B\'eranger}
\affiliation{%
  \institution{Inria Paris-Rocquencourt}
  \city{Rocquencourt}
  \country{France}
}

\author{Philip Lundrigan}
\affiliation{%
  \institution{Brigham Young University}
  \city{Provo}
  \state{Utah}
  \country{USA}
}
\email{lundrigan@byu.edu}

%%
%% By default, the full list of authors will be used in the page
%% headers. Often, this list is too long, and will overlap
%% other information printed in the page headers. This command allows
%% the author to define a more concise list
%% of authors' names for this purpose.
% \renewcommand{\shortauthors}{Trovato et al.}

%%
%% The abstract is a short summary of the work to be presented in the
%% article.
\begin{abstract}
Lorem ipsum dolor sit amet. Aut maxime quos in odio totam et voluptate consequatur et incidunt nisi ut velit voluptas aut corrupti voluptatibus? Qui optio ipsam id doloremque suscipit et magni enim.
Qui velit voluptates eos consequatur tempore et nisi porro. Est quia doloribus ex sint debitis rem quasi dignissimos est eligendi reprehenderit. Aut eveniet rerum aut rerum voluptas et dolores maxime.
Sed ipsa numquam eos galisum laborum eum explicabo fuga rem molestiae aliquid eum laudantium natus. Et sint nostrum quo quos quia non quam nemo in repellat exercitationem et voluptas harum ea repellendus rerum. Ea quia facilis et ullam distinctio ea numquam unde aut nostrum odit aut dolorem sunt qui numquam quia.
\end{abstract}

%%
%% The code below is generated by the tool at http://dl.acm.org/ccs.cfm.
%% Please copy and paste the code instead of the example below.
%%
% \begin{CCSXML}
% <ccs2012>
%  <concept>
%   <concept_id>00000000.0000000.0000000</concept_id>
%   <concept_desc>Do Not Use This Code, Generate the Correct Terms for Your Paper</concept_desc>
%   <concept_significance>500</concept_significance>
%  </concept>
%  <concept>
%   <concept_id>00000000.00000000.00000000</concept_id>
%   <concept_desc>Do Not Use This Code, Generate the Correct Terms for Your Paper</concept_desc>
%   <concept_significance>300</concept_significance>
%  </concept>
%  <concept>
%   <concept_id>00000000.00000000.00000000</concept_id>
%   <concept_desc>Do Not Use This Code, Generate the Correct Terms for Your Paper</concept_desc>
%   <concept_significance>100</concept_significance>
%  </concept>
%  <concept>
%   <concept_id>00000000.00000000.00000000</concept_id>
%   <concept_desc>Do Not Use This Code, Generate the Correct Terms for Your Paper</concept_desc>
%   <concept_significance>100</concept_significance>
%  </concept>
% </ccs2012>
% \end{CCSXML}

% \ccsdesc[500]{Do Not Use This Code~Generate the Correct Terms for Your Paper}
% \ccsdesc[300]{Do Not Use This Code~Generate the Correct Terms for Your Paper}
% \ccsdesc{Do Not Use This Code~Generate the Correct Terms for Your Paper}
% \ccsdesc[100]{Do Not Use This Code~Generate the Correct Terms for Your Paper}

%%
%% Keywords. The author(s) should pick words that accurately describe
%% the work being presented. Separate the keywords with commas.
% \keywords{yes, no, maybe}

\received{20 February 2007}
\received[revised]{12 March 2009}
\received[accepted]{5 June 2009}

%%
%% This command processes the author and affiliation and title
%% information and builds the first part of the formatted document.
\maketitle

\section{Introduction}

Transportation is essential to modern society, with the movement of people and goods driving economic and social activity.
In 2019, it was estimated that more than 70\% of US freight was transported by the trucking industry~\cite{bishop_utah_2022}.
Diesel engines are a major source of harmful pollutants including Particulate Matter (PM) and Nitrogen Oxides (NOx), which are harmful to human health and the environment.
For example, short-term exposure to diesel motor emissions can cause acute irritation and asthma-like symptoms, while long-term exposure is linked to increased mortality and lung cancer~\cite{wichmann_diesel_2007}.
Other studies have linked traffic-related air pollution to increased risk of neurological conditions, including depression, anxiety, and dementia~\cite{miner_car_2024}.
On-road vehicle emissions also damage our natural environment, contributing to global climate change~\cite{nat_geo_2025, anenberg_global_2019}

Reducing on-road diesel emissions is therefore an important public health and environmental goal.
Researchers have noted the large impact that targeting high-emitting vehicles can have on reducing overall emissions, noting that a small fraction of vehicles is responsible for a disproportionate amount of pollution.~\cite{CARB_2015, ban-weiss_measurement_2009, shen_evaluation_2022}
Conventional strategies to reduce on-road pollution from diesel trucks include aftertreatment systems, onboard sensors, and government regulatory Inspection and Maintenance (I/M) programs.
Aftertreatment systems significantly reduce NOx and PM emissions from new diesel vehicles, but most high-emitting vehicles are likely older or broken trucks with ineffective aftertreatment.
I/M programs are designed to identify high-emitting vehicles through annual inspections, but researchers have questioned the effectiveness of such programs, showing that areas with I/M programs do not have a statistically significant decrease in emissions compared to areas without I/M programs~\cite{maricq_extreme_2025, bishop_inspection_2020}.
Roadside emission monitoring systems provide an alternative approach to identifying high-emitting vehicles, using sensors placed along roadways to measure emissions from passing vehicles.


\subsection{Low-Cost Sensors}

While roadside emission monitoring has been the focus of many research teams for estimating pollution from on-road vehicles~\cite{burgard_spectroscopy_2006, watne_fresh_2018, liu_roadside_2019, sugrue_comparing_2020, shen_evaluation_2022}, it has not yet been adopted for large-scale detection of high emitters.
The main challenges are the high cost and complexity of such systems, with expensive sensing instruments that are often tens of thousands of dollars, and the need for onsite personel for operation and maintenance, and calibration.
In this paper, we explore the use of low-cost sensors for roadside emission monitoring, which could lead to a cost-effective widely deployable roadside emission monitorying system.

Many studies have evaluated the quality of low-cost air quality sensors by comparing them to high-cost reference instruments. 
Although low-cost sensors often follow a similar trend to high-costs sensors, they have been found to be less accurate, with significant measurement error and variability from sensor to sensor.
The high-cost sensors often draw air into an internal chamber for analysis, low-cost sensors perform analysis on ambient air and are effected by variability of environmental conditions such as temperature, pressure, and air flow rate.
Additionally, many lab-grad gas analyzers sample rates of up to 10 Hz, while most low-cost sensors can only be sampled at approximately 1 Hz or less.
Low-cost sensors are designed and most often used for \textit{ambient sensing}, where the primary concern is measurement levels over the course of minutes, hours, or days.

Despite these limitations, low-cost sensors still might the potential to identify high-emitting vehicles, even if the absolute measurments are not exact.
TODO: Mention Sugrue et al and shen et al and what they found about low-cost sensors for roadside emission monitoring.
We design and deploy a roadside diesel emission monitoring system using low- and high-cost sensors, enabling a direct comparison of the two sensor types for estimating vehicle emission rates in a laboratory and real-world setting.
We also explore the use of machine learning to enhance the low-cost sensor data, with the goal of improving the accuracy of emission rate estimates from low-cost sensors.

\section{Methodology}

\subsection{Data Collection}

We implemented a low-cost data acquisition pipeline for reliable data collection, storage, processesing, and visualization.
We collected data from up to 8 low-cost sensors and 4 high-cost sensors simultatneously.
Table 1 summarizes the sensors used in this study.

\begin{figure*}[b]
  \centering
  \includegraphics[width=\textwidth]{figures/pipeline_diagram.jpg}
  \caption{diagram}
  \Description{description}
\end{figure*}

We interfaced with the low-cost sensors using an ESP32 microcontroller that communicated with sensors over I2C and UART protocols.
The high-cost sensors were read using a Raspberry Pi, communicating with the UART protocol over USB serial connections.
After the data analyzed in this papers was collected, we upgraded the low-cost data system to also use a Raspberry Pi microcomputer for more reliable data collection and easier interfacing with sensors.

Data was transmitted from these data aquisition devices over the local wifi network in the labratory, and using Starlink internet connection in the field deployment.
We used the MQTT pup-sub protocol, with Mosquitto MQTT broker running on a local server.
Python scripts listened to MQTT topics an stored incoming data into our postgreSQL database.
We used TimescaleDB, a postgreSQL extension for time-series data, to effieciently store our sensor data.
We also used Grafana to create dashboards for data visualization, allowing for real-time monitoring during data collection.
We set up a website on our local server to provide easy access to Grafana and out database.
We also created a page to easily record experiment start and end times in the database and export data in a time-aligned CSV format.

Air was sampled through an inlet near the engine exhaust and drawn through tubing to be distributed to the high and low-cost sensors.
In the field, we had one inlet above the road to sample trucks with high exhausts, and one inlet in a speedbump to sample trucks with low exhausts. 
The low-cost sensors were housed in a metal canister with inlets and outlets allowing them to receive a controlled sample of exaust similar to the high-cost sensors.
Electrical connection to the low-cost sensors were passed into the canister through an airtight bulkhead connector.


\begin{figure}[htbp]
	\centering
  \includegraphics[width=0.9\linewidth]{figures/roadside.jpg}
	\caption{canister}
  \Description{description2}
	\label{fig:bulkhead}
\end{figure}

\subsection{Emission Rates}

\begin{itemize}
\item We calculate emission rates with some equations
\item We automate peak detection and rate calculation process with some python scripts
\end{itemize}

\subsection{Machine Learning}

\begin{itemize}
\item We improve low-cost sensor data with LSTM machine learning model
\item This could make low-cost sensors more viable for emission monitoring
\item People have used machine learning for low-cost sensor calibration, but it is all for long term applications with data sampled every few minues or every hour
\item We use machine learning to improve small windows of data where a truck passes, with data sampled every 2 seconds
\end{itemize}

\section{Results}

\subsection{Lab Testing}

\begin{itemize}
\item We simulated trucks driving by in a lab setting
\item The spikes look pretty good, but low-cost sensor emission rates 
\item Using machine learning improves the low-cost sensor data very well
\item Improves emission rates a bit but not amazing
\end{itemize}

\subsection{Deployment}

\begin{itemize}
\item We brought low- and high-cost sensors to a real world deployment for 3 days
\item Data was much worse in the real world
\item Machine learning helped a bit, but not as much as in the lab
\end{itemize}


\section{Conclusion}

\begin{itemize}
  \item Overall, we built a cool systems
  \item The low-cost sensors show promise, but depend and getting a good sample, and are inconsistent
  \item Machine Learning shows potential, but needs better data and further analysis and transferability
\end{itemize}


%%
%% The next two lines define the bibliography style to be used, and
%% the bibliography file.
\bibliographystyle{ACM-Reference-Format}
\bibliography{sample-base}


\end{document}
\endinput
%%
%% End of file `sample-sigconf.tex'.
